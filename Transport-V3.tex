\documentclass[reprint,aip,apl,floatfix,linenumbers,superscriptaddress]{revtex4-1}
\usepackage[product-units = power]{siunitx}
\usepackage[version=3]{mhchem}
\usepackage[utf8]{inputenc}
\usepackage{natbib}

\DeclareSIUnit\BohrMagneton{$\mu_B$}
\DeclareSIUnit\formulaunit{f.u.}
\DeclareSIUnit\atomicunit{a.u.}
\DeclareSIUnit\arbunit{arb.unit}
\DeclareSIUnit\torr{Torr}

\usepackage{pgfplots}
\usepackage{tikz}
\usepackage{pdfpages}

\pgfplotsset{small,width=\columnwidth, every axis/.append style={ylabel near ticks, thick, tick style={semithick}}}
\usetikzlibrary{backgrounds}
\usepgfplotslibrary{groupplots}
\usetikzlibrary{positioning}
%\usetikzlibrary{external}
\usepgfplotslibrary{external}
\tikzsetexternalprefix{figures/}
\tikzexternalize
%\tikzexternalize[shell escape=-enable-write18]

\begin{document}

%title, authors, affiliations
\title{Transport properties of cubic zero-moment ferromagnetic \ce{Mn2Ru$_x$Ga} thin films}
\author{Naganivetha Thiyagarajah}
\author{Yong-Chang Lau}
\author{Karsten Rode}
\author{Davide Betto}
\author{Kiril Borisov}
\author{M. Venkatesan}
\author{J. M. D. Coey}
\author{Plamen Stamenov}
\affiliation{CRANN, AMBER and School of Physics, Trinity College Dublin, Dublin 2, Ireland}

\date{\today}

%abstract needs to be written
\begin{abstract}
The spin-dependent transport properties of cubic \ce{Mn2Ru$_x$Ga} thin-films are studied as a function of the the Ru concentration, $x$ and the substrate induced strain. We find that at Ru concentration $x\approx\num{0.7}$, which shows practically zero magnetization, the spontaneous Hall effect at room temperature reverses sign and the spontaneous Hall angle is maximized. In addition, a small tetragonal distortion, $c/a\sim\num{2}\%$, allows us to tune the compensation of the two Mn sub-lattices to a preferred temperature at, above or below room temperature. Having two handles on the zero moment half magnetic properties of \ce{Mn2Ru$_x$Ga} opens up the possibilities for using this new class of material in various spintronic devices. We also present the initial work on magnetoresistive devices based on pseudo-spin-valves with \ce{Mn2Ru$_x$Ga} electrodes.

\end{abstract}
\maketitle

%Introduction 
\section{Introduction}
\label{sec:intro}

Cubic ferromagnetic Heusler compounds are a family of magnetic materials that often exhibit higher spin polarization at the Fermi level than binary ferromagnetic $3d$ alloys\cite{Graf2013}. Some of the materials are half-metals with a gap in the spin-polarized density of states for one spin band which should make them ideal candidates for spin-valves or MTJs\cite{PhysRevB.28.1745,Wang2009,Takahashi2011,Tsunegi2008}.  Since the prediction by van Leuken and de Groot in $1995$, of a half-metallic material with two in-equivalent magnetic sub-lattices whose moments cancel out \cite{PhysRevLett.50.2024}, researchers have worked on fabricating such a material. While electronic structure calculations predicted several such compounds\cite{Wurmehl2006, Hu2012, Galanakis2006}, fabrication of such materials had failed\cite{Hu2012,PhysRevB.79.100406}. In $2014$, Kurt \textit{et. al.} reported the growth of thin films of \ce{Mn2Ru$_x$Ga} (\ce{MRG}), which was identified as a zero-moment ferrimagnet with high spin polarization and showed evidence of half-metallicity\cite{KurtPRL2014}. 

Here we report on the temperature, composition and thickness dependent transport properties of \ce{MRG}, which are at or near compensation point ($0.6<x<1.1$). Addition of Ru to the cubic \ce{Mn2Ga} structure provides both states (\num{12}) and electrons (\num{8}). Based on the on the empirical Slater-Pauling rules, should result in perfect compensation for \ce{Mn2Ru$_{0.5}$Ga}. However the addition of Ru is likely to change both the shape and position of the Mn bands leading to a more complex behaviour of the magnetic and spin-dependent transport properties. In addition the tetragonal distortion ($c/a$) can also affect the band structure, hence we also look at strain as a possible control parameter in engineering the \ce{MRG} fully compensated half metallic system.


%experiment 
\section{Experimental techniques}
\label{sec:exp_tech}

\ce{MRG} films of thickness \SIrange{4}{70}{\nano\metre} were grown on \ce{MgO} (001) substrates by dc-magnetron sputtering at \SI{250}{\celsius} substrate temperature and base pressure \SI{2e-8}{\torr} in a Shamrock deposition system. The films were co-sputtered from a Mn2Ga target and Ru target, and the Ru composition was controlled by keeping the \ce{Mn2Ga} sputtering power fixed while varying that of Ru. The MRG films were capped with a  $\sim \SI{2}{\nano\metre}$ \ce{Al2O3} layer to prevent oxidation.  The crystal structure and lattice parameters were determined by $2\theta-\theta$ and reciprocal space map (RSM) scans using a BRUKER D8 diffractometer. In order to determine the \ce{Ru} concentration $x$, we deposited four samples with varying \ce{Mn2Ga} target power along with a \ce{Ru} film. The density and thickness of the samples were then measured using x-ray reflectivity. Based on the measured density and lattice parameters of these 5 control samples, we establish a relation between the x-ray density and the \ce{Ru} concentration $x$ against which all the samples are calibrated. Magnetization measurements were made using a Quantum Design superconducting quantum interference device (SQUID) magnetometer. The transport measurements were conducted on unpatterned \ce{MRG} films in a physical properties measurement system (PPMS) for temperatures from \SIrange{10}{400}{\kelvin}. The maximum applied magnetic fields, $\mu_0H$, for the two systems were \SI{5}{\tesla} and \SI{14}{\tesla} respectively. A summary of sample properties is provided in Table \ref{tab:samples}. We also incorporated the \ce{MRG} as the hard layer into a pseudo-spin-valve with the structure, \ce{MgO}/ \ce{MRG}(15)/\ce{Cu}(2.8)/[\ce{Co}(0.2)/\ce{Pd}(0.6)]$_6$/\ce{Ta}(\SI{3}{\nano\metre}) in order to investigate the spin dependent transport. The \ce{MRG} layer was grown at \SI{250}{\celsius}, then cooled down to room temperature, and was subsequently transferred to a different deposition chamber for the \ce{Cu}/[\ce{Co}/\ce{Pd}] multilayer deposition. Atomic force microscopy measurements of the \ce{MRG} film showed a roughness of $\sim \SI{0.2}{\nano\metre}$, free of pinholes. 

%table with properties
\begingroup
\squeezetable
\begin{table}
  \caption{Summary of sample properties. The temperature at which full compensation occurs, $T_{comp}$ was defined by the temperature where $\partial \rho_{xy}/\partial T$ reaches its maximum.}
  \begin{ruledtabular}
  \begin{tabular}{c c c c c }
    \ce{Ru} $x$ & $t$ &  $c/a-1$ & $M_s$ & $T_{comp}$ \\ 
		 & \SI{}{\nano\metre} & \SI{}{\percent} & \SI{}{\BohrMagneton}  & \SI{}{\kelvin} \\
     \hline
     \num{0.62} & \num{70} & \num{2.07} & \num{-0.09} & \numrange[range-phrase = --]{100}{200} \\ %NT059
     \num{0.69} & \num{70} & \num{1.76} & \num{0.03} & \numrange[range-phrase = --]{200}{300} \\ %NT058
     \num{0.73} & \num{70} & \num{1.83} & \num{0.07} & \numrange[range-phrase = --]{300}{360} \\ %NT063
		 \num{0.77} & \num{70} & \num{1.92} & \num{0.09} & \num{> 360} \\ %NT055
		 \num{1.09} & \num{70} & \num{1.82} & \num{0.07} & \num{> 360} \\ %NT046
     \num{1.12} & \num{70} & \num{1.84} & \num{0.07} & \num{387} \\ %NT072 
		 \num{1.01} & \num{34} & \num{1.92} & - & \num{335} \\ %NT073
		 \num{0.98} & \num{16} & \num{2.24} & - & \num{280} \\ %NT074
     \num{1.09} & \num{8} & \num{2.90} & - & \num{214} \\ %NT075
		 \num{1.07} & \num{4} & \num{3.60} & - & \num{< 10} \\ %NT076	 
  \end{tabular}
  \end{ruledtabular}
  \label{tab:samples}
\end{table}
\endgroup

%results section needs to be expanded - more discussion
\section{Results and Discussion}
\label{sec:results_discussion}

%fig 1 - XRD - some tweaking
\begin{figure}
\begin{tikzpicture}
	\begin{axis}[xshift=0.4\columnwidth, yshift=3.75cm, xlabel = thickness (\SI{}{\nano\metre}), ylabel = c (\SI{}{\angstrom}),xlabel near ticks, tiny]
		\addplot table {./data/cvst.txt};
	\end{axis}
	\begin{axis}[xmin=39,xmax=64,ymin=2,ymax=10000,ymode=log,xlabel=$2\theta$ (\SI{}{\degree}),ylabel=Intensity (\SI{}{\arbunit})]
		\addplot[domain=39:64,black,very thick] table[x expr=\thisrowno{0},y expr = \thisrowno{2}]{./data/NT072-HA.txt};
		\addplot[domain=39:64,red,very thick] table[x expr=\thisrowno{0},y expr = \thisrowno{2}]{./data/NT073-HA.txt};
		\addplot[domain=39:64,blue,very thick] table[x expr=\thisrowno{0},y expr = \thisrowno{2}]{./data/NT074-HA.txt};
		\addplot[domain=39:64,green,very thick] table[x expr=\thisrowno{0},y expr = \thisrowno{2}]{./data/NT075-HA.txt};
		\addplot[domain=39:64,orange,very thick] table[x expr=\thisrowno{0},y expr = \thisrowno{2}]{./data/NT076-HA.txt};
		%\node at (axis cs:31,2000) [rotate=90] {MRG (002)};
		\node at (axis cs:41,2000) [rotate=90] {MgO (002)};
		\node at (axis cs:62.5,2000) [rotate=90] {MRG (004)};
	\end{axis}
\end{tikzpicture}
\caption{XRD of thin films of \ce{Mn2Ru$_x$Ga} of thickness from \SIrange{70}{4}{\nano\metre} grown on \ce{MgO} substrates. Inset shows the dependence of the out-of-plane lattice parameter (c) on the thickness of the film, indicating that the substrate induced strain is increasingly relaxed as the thickness increases.}
\label{fig:xrd}
\end{figure}

The crystal structure of the cubic \ce{MRG} films with different thickness and compositions were probed using $2\theta-\theta$ x-ray diffraction (XRD) as shown in Fig. \ref{fig:xrd}. The out-of-plane lattice parameter, $c$, is between \SI{0.598}{\nano\metre} and \SI{0.618}{\nano\metre}, depending on the \ce{Ru} concentration and film thickness (insert of Fig. \ref{fig:xrd}). The in-plane lattice parameter, $a$, determined from reciprocal space maps was found to be \SI{0.596}{\nano\metre} for all samples, which is precisely matched to that of the \ce{MgO} substrate ($\sqrt{2} a_0\left(\ce{MgO} \right) = \SI{0.5956}{\nano\metre}$). This confirms the cubic nature of the \ce{MRG} films with a slight tetragonal out-of-plane distortion ($c/a-1$ between \num{1.8}\% and \num{3.6}\%).
%fig 2 - SQUID ok - is it needed? 
\begin{figure}
\begin{tikzpicture}
	\begin{axis}[grid=both, xmin=-5,xmax=5,ymin=-15,ymax=15,xlabel=$\mu_0H$ (\SI{}{\tesla}),ylabel = Moment (\SI{}{\kilo\ampere\per\metre}),legend pos = north west]
		\addplot+[mark=+,black, smooth] table [x expr = \thisrowno{0},y expr = \thisrowno{1}/1000]{./data/squid.txt};
		\addplot+[mark=x,red, smooth] table [x expr = \thisrowno{2},y expr = \thisrowno{3}/1000]{./data/squid.txt};
		\legend{in-plane,out-of-plane}	
	\end{axis}
\end{tikzpicture}
\caption{In-plane and out-of-plane magnetization loops of \ce{Mn2Ru$_x$Ga} sample of thickness \SI{70}{\nano\metre}, measured in a SQUID magnetometer at \SI{300}{\kelvin}.}
\label{fig:squid}
\end{figure}

%fig 3 - ehe curves of Ru_conc and temp dependence? 
\begin{figure}
\begin{tikzpicture}
	\begin{axis}[name = plotru,xmin=-2,xmax=2,ymin=-0.6,ymax=0.6,ylabel = $R_{XY}$ (\SI{}{\ohm}),legend style={legend pos=south west}]
		\addplot+[mark=none,black,dotted,very thick] table {./data/NT046EHE.txt};
		\addplot+[mark=none,red,dashed,very thick] table {./data/NT055EHE.txt};
		\addplot+[mark=none,blue,dashdotted,very thick] table {./data/NT063EHE.txt};
		\addplot+[mark=none,green,densely dotted,very thick] table {./data/NT058EHE.txt};
		\addplot+[mark=none,orange,solid,very thick] table {./data/NT059EHE.txt};
		\node at (axis cs:-1.75,0.55) {(a)};
		\legend{$x=1.09$,$x=0.77$,$x=0.73$,$x=0.69$,$x=0.62$}	
	\end{axis}
	\begin{axis}[name = plott,at={($(plotru.south)-(0,0.5cm)$)},anchor=north, xmin=-12,xmax=12,ymin=-0.6,ylabel = $R_{XY}$ (\SI{}{\ohm}), ymax=0.6,xlabel=$\mu_0H$ (\SI{}{\tesla}),legend style={legend pos=south west}]
		\addplot+[mark=none,black,dotted,very thick] table [x expr =\thisrowno{0}, y expr =\thisrowno{1}]{./data/400K.txt};
		\addplot+[mark=none,red,dashed,very thick] table [x expr =\thisrowno{0}, y expr =\thisrowno{1}]{./data/350K.txt};
		\addplot+[mark=none,blue,dashdotted,very thick] table [x expr =\thisrowno{0}, y expr =\thisrowno{1}]{./data/300K.txt};
		\addplot+[mark=none,green,densely dotted,very thick] table [x expr =\thisrowno{0}, y expr =\thisrowno{1}]{./data/200K.txt};
		\addplot+[mark=none,orange,solid,very thick] table [x expr =\thisrowno{0}, y expr = \thisrowno{1}]{./data/100K.txt};
		\node at (axis cs:-11,0.55) {(b)};
		\legend{\SI{400}{\kelvin},\SI{350}{\kelvin},\SI{300}{\kelvin},\SI{200}{\kelvin},\SI{100}{\kelvin}}	
	\end{axis}
\end{tikzpicture}
\caption{SHE loops measured of \ce{Mn2Ru$_x$Ga} for (a) various Ru compositions ($0.6<x<1.1$) and (b) temperatures between \SI{10}{\kelvin} and \SI{400}{\kelvin}, which illustrates the change of sign of the spontaneous hall coefficient between $x=0.62$ and $x=0.73$ and \SI{300}{\kelvin} and \SI{350}{\kelvin} respectively.}
\label{fig:she}
\end{figure}

%figure4 - Ru conc ms and ehe params
\begin{figure}
\begin{tikzpicture}
\begin{axis}[name=plot1,height=4cm,xmin=0.6, xmax=1.1,ymin=-0.15,ymax=0.3, xticklabels={,,}, ylabel = $M_s$ (\SI{}{\BohrMagneton\per\formulaunit})]
		\addplot+[mark=square,black,only marks, mark size = 3] table [x expr=\thisrowno{0},y expr = \thisrowno{1}]{./data/ruconc.txt}; 
		\addplot+[mark=none,black,smooth,very thick] table {./data/fitlin.txt};
		\node at (axis cs:0.625,0.21) {(a)};
\end{axis}
\begin{axis}[name=plot2,at={($(plot1.south)$)},anchor=north,height=4cm,xmin=0.6, xmax=1.1,ymin=0,ymax=2.2, hide x axis,  axis y line*=left, ylabel = $\mu_0H_c$ (\SI{}{\tesla})]
		\addplot+[mark=o,blue,smooth, tension=0.001, mark size = 3, very thick] table [x expr=\thisrowno{0},y expr = \thisrowno{2}]{./data/ruconc3.txt}; 	
		\node at (axis cs:0.625,2) {(a)};
\end{axis}
\begin{axis}[name=plot3,at={($(plot1.south)$)},anchor=north,height=4cm,xmin=0.6, xmax=1.1,ymin=-6,ymax=6, axis y line*=right, xlabel = Ru concentration ($x$), ylabel = SHA (\%)]
		\addplot+[mark=triangle,red,smooth, tension=0.001, mark size = 3, very thick] table [x expr=\thisrowno{0},y expr = \thisrowno{3}]{./data/ruconc3.txt}; 	
\end{axis}
\end{tikzpicture}
\caption{(a) Extracted magnetization at \SI{300}{\kelvin} (in \SI{}{\BohrMagneton\per\formulaunit}), for samples of thickness \SI{70}{\nano\metre} with different \ce{Ru} composition ($0.6<x<1.1$). The change of sign of the magnetization was established by SHE sign reversal at compensation. (b) Coercive field  and spontaneous Hall angle as a function of \ce{Ru} composition, extracted from SHE measurements carried out at \SI{300}{\kelvin}, for the same \ce{MRG} samples as in (a).}
\label{fig:ru_conc}
\end{figure}

Fig. \ref{fig:squid} shows the magnetization measurement at \SI{300}{\kelvin} of a typical \ce{MRG} film of \SI{70}{\nano\metre} near compensation of the magnetic sub-lattices. Clear out-of-plane anisotropy with a large coercivity of \SI{1.2}{\tesla} is evident.  A small soft in-plane component is also clearly visible. As the \ce{Ru} concentration is reduced from $x=\num{1.09}$, the magnetization reduces, until it falls practically to zero (\SI{12}{\kilo\ampere\per\metre} or \SI{0.07}{\BohrMagneton\per\formulaunit}) at $x=\num{0.68}$ as shown in Fig. \ref{fig:ru_conc}(a). We can attribute this to the almost perfect compensation of the two \ce{Mn} sub-lattices at room temperature. On further reduction of Ru the magnetization again increases. We denote this as a negative magnetization, coincident with the reversal in sign in the room temperature spontaneous Hall effect (SHE) measurements as shown in Fig. \ref{fig:she}(a). From the SHE measurements with varying Ru content, we extracted the coercivity, $\mu_0H_c$, and spontaneous Hall angle (SHA) (defined as $\rho_H$/$\rho$) (Fig. \ref{fig:ru_conc}(b)). As the magnetization approaches zero the coercivity clearly diverges (the sample closest to compensation at room temperature could not be saturated at an applied field of \SI{5}{\tesla}). The recorded SHA for samples near compensation ($\sim \num{5}\%$) are about a magnitude larger than those reported for other 3d ferromagnets at room temperature (\numrange{0.2}{0.3}\%)\cite{dorleijn1976} and comparable to SHA recorded for amorphous rare earth transition metal alloys\cite{Kim2001}. A high SHA is indicative of much lower carrier concentrations and a high spin polarization.

%fig 5 on strain/thickness goes here
\begin{figure}
\begin{tikzpicture}
	\begin{axis} [name=plota,height=5cm,xmin=0,xmax=400,ymin=-0.005,ymax=0.05,xticklabels={,,}, ylabel = $\delta R_{XY}/\delta T$ (\SI{}{\ohm\per\kelvin}),legend style={legend pos=north west,font=\tiny},]
		\addplot+[mark=none,black,solid,smooth, very thick] table [x expr = \thisrowno{0}, y expr = \thisrowno{1}]{./data/rvst.txt};
		\addplot+[mark=none,red,dotted,smooth, very thick] table [x expr = \thisrowno{2}, y expr = \thisrowno{3}]{./data/rvst.txt};
		\addplot+[mark=none,blue,dashed,smooth, very thick] table [x expr = \thisrowno{4}, y expr = \thisrowno{5}]{./data/rvst.txt};
		\addplot+[mark=none,green,densely dotted,smooth, very thick] table [x expr = \thisrowno{6}, y expr = \thisrowno{7}]{./data/rvst.txt};
		%\node at (axis cs:235,0.025) {$T_M=\SI{235}{\kelvin}$};
		%\node at (axis cs:278,0.031) {$T_M=\SI{278}{\kelvin}$};
		%\node at (axis cs:335,0.045) {$T_M=\SI{335}{\kelvin}$};
		%\node at (axis cs:387,0.025) {$T_M=\SI{387}{\kelvin}$};	
		\legend{\SI{70}{\nano\metre},\SI{34}{\nano\metre},\SI{16}{\nano\metre},\SI{8}{\nano\metre}}	
	\end{axis}
	\begin{axis} [name=plotb,at={($(plota.south)$)},anchor=north,height=5cm,xmin=0,xmax=400,ymin=-0.5,ymax=10,xticklabels={,,}, ylabel = $\mu_0H_C$ (\SI{}{\tesla}),legend style={legend pos=north west,font=\tiny}]
		\addplot+[mark=o,black,smooth, very thick, mark size=2,unbounded coords=discard] table [x expr = \thisrowno{0}, y expr = \thisrowno{1}/10]{./data/Hct.txt};
		\addplot+[mark=square,red,smooth, very thick,mark size=2,unbounded coords=discard] table [x expr = \thisrowno{0}, y expr = \thisrowno{2}/10]{./data/Hct.txt};
		\addplot+[mark=triangle,blue,smooth, very thick,mark size=2,unbounded coords=discard] table [x expr = \thisrowno{0}, y expr = \thisrowno{3}/10]{./data/Hct.txt};
		\addplot+[mark=+,green,,smooth, very thick,mark size=2,unbounded coords=discard] table [x expr = \thisrowno{0}, y expr = \thisrowno{4}/10]{./data/Hct.txt};
		\addplot+[mark=x,orange,smooth, very thick,mark size=2,unbounded coords=discard] table [x expr = \thisrowno{0}, y expr = \thisrowno{5}/10]{./data/Hct.txt};
		\legend{\SI{70}{\nano\metre},\SI{34}{\nano\metre},\SI{16}{\nano\metre},\SI{8}{\nano\metre},\SI{4}{\nano\metre}}	
	\end{axis}
	\begin{axis} [name=plotc,at={($(plotb.south)$)},anchor=north,height=5cm,xmin=0,xmax=400,ymin=-2,ymax=2.2,xlabel= Temperature (\SI{}{\kelvin}), ylabel = SHA (\%),legend style={legend pos=north west,font=\tiny}]
		\addplot+[mark=o,black,smooth, very thick, mark size=2,unbounded coords=discard] table [x expr = \thisrowno{0}, y expr = \thisrowno{1}]{./data/SHAt.txt};
		\addplot+[mark=square,red,smooth, very thick,mark size=2,unbounded coords=discard] table [x expr = \thisrowno{0}, y expr = \thisrowno{2}]{./data/SHAt.txt};
		\addplot+[mark=triangle,blue,smooth, very thick,mark size=2,unbounded coords=discard] table [x expr = \thisrowno{0}, y expr = \thisrowno{3}]{./data/SHAt.txt};
		\addplot+[mark=+,green,,smooth, very thick,mark size=2,unbounded coords=discard] table [x expr = \thisrowno{0}, y expr = \thisrowno{4}]{./data/SHAt.txt};
		\addplot+[mark=x,orange,smooth, very thick,mark size=2,unbounded coords=discard] table [x expr = \thisrowno{0}, y expr = \thisrowno{5}]{./data/SHAt.txt};
		%\legend{\SI{70}{\nano\metre},\SI{34}{\nano\metre},\SI{16}{\nano\metre},\SI{8}{\nano\metre},\SI{4}{\nano\metre}}	
	\end{axis}
\end{tikzpicture}
\caption{(a) Variation of compensation temperature with the thickness of \ce{MRG} film of same \ce{Ru} concentration, given by the derivative of the resistance w.r.t temperature. The compensation temperature shifts to lower temperatures with decreasing thickness. (b) Extracted coercive field and (c) spontaneous Hall angle as a function of temperature for samples with the same \ce{Ru} concentration ($x \sim \num{1.0}$) and various thickness from \SIrange{70}{4}{\nano\metre}.}
\label{fig:strain}
\end{figure}

As shown in Fig. \ref{fig:xrd}, the \ce{MRG} films are increasingly strained as the thickness of the film is reduced. It has been predicted that the magnetization may depend strongly on the lattice distortion since this would have an effect on the interaction between neighbouring  atoms. We prepared \ce{MRG} samples of different thickness from \SI{70}{\nano\metre} down to \SI{4}{\nano\metre} and measured their SHE response at different temperatures from \SI{400}{\kelvin} to \SI{4}{\kelvin} in the PPMS. Fig. \ref{fig:she}(b) shows a typical SHE response over the temperature range for the sample of \SI{34}{\nano\metre} thickness. It can be seen that the coercivity diverges to $\sim\SI{9}{\tesla}$ at \SI{350}{\kelvin} and the sign of the SHE loop reverses at \SI{300}{\kelvin}. This indicates that the compensation temperature lies between \SI{300}{\kelvin} and \SI{350}{\kelvin}.  By plotting the derivative of the Hall resistance w.r.t temperature, $\delta R_{XY}/\delta T$, as shown in Fig. \ref{fig:strain}(a), it can be seen that this compensation temperature shifts to lower temperatures as the thickness of the \ce{MRG} is reduced. It is worth noting that the compensation temperature varies with both the Ru content and strain. Since the compensation is achieved by the cancelling out of the moment of the two inequivalent \ce{Mn} sub-lattices, this shift in compensation temperature may be due to the slightly different temperature dependence of the two sub-lattices. As with samples with different \ce{Ru} content, the extracted coercivity and SHA show maximum values near the compensation temperature for each thickness as shown in Fig. \ref{fig:strain}(b) and (c) respectively.

%fig 6 GMR goes here (six figures seems way too many!)
\begin{figure}
\begin{tikzpicture}
	\begin{axis}[xmin=-5,xmax=5,ymin=0,ymax=1.6, xlabel=$\mu_0H$ (\SI{}{\tesla}),ylabel=MR (\%)]
		\addplot[mark=none,black,very thick] table[x expr=\thisrowno{0}/10000,y expr = \thisrowno{1}]{./data/gmr2k.txt};
		%\addplot[domain=-50000:50000,red,very thick] table[x expr=\thisrowno{0}/10000,y expr = \thisrowno{1}+0.5]{./data/gmr4k.txt};
		\addplot[mark=none,blue,very thick] table[x expr=\thisrowno{0}/10000,y expr = \thisrowno{1}+0.2]{./data/gmr25k.txt};
		%\addplot[mark=none,green,very thick] table[x expr=\thisrowno{0}/10000,y expr = \thisrowno{1}+1.5]{./data/gmr50k.txt};
		%\addplot[mark=none,orange,very thick] table[x expr=\thisrowno{0}/10000,y expr = \thisrowno{1}+2.0]{./data/gmr75k.txt};
		\addplot[mark=none,red,very thick] table[x expr=\thisrowno{0}/10000,y expr = \thisrowno{1}+0.4]{./data/gmr100k.txt};
		%\addplot[mark=none,cyan,very thick] table[x expr=\thisrowno{0}/10000,y expr = \thisrowno{1}+3.0]{./data/gmr150k.txt};
		\addplot[mark=none,green,very thick] table[x expr=\thisrowno{0}/10000,y expr = \thisrowno{1}+0.6]{./data/gmr200k.txt};
		%\addplot[mark=none,teal,very thick] table[x expr=\thisrowno{0}/10000,y expr = \thisrowno{1}+4.0]{./data/gmr300k.txt};
		\node at (axis cs:-4.5,0.15) {\SI{2}{\kelvin}};
		\node at (axis cs:-4.5,0.35) {\SI{25}{\kelvin}};
		\node at (axis cs:-4.5,0.55) {\SI{100}{\kelvin}};
		\node at (axis cs:-4.5,0.75) {\SI{200}{\kelvin}};	
	\end{axis}
	\begin{axis}[xshift=0.47\columnwidth, yshift=3.75cm, width= 3cm, yticklabels={0,0.05,0.1,0.15}, xlabel = T (\SI{}{\kelvin}), ylabel = GMR (\%),xlabel near ticks, tiny, xmin=0, xmax=300, ymin = 0, ymax = 0.15] %/tikz/background rectangle/.style={fill=white}, show background rectangle
		\addplot [mark=*, only marks, mark size =2] table {./data/gmr.txt};
		\addplot [mark=none, red, smooth, very thick] table {./data/gmrfit.txt};
	\end{axis}
\end{tikzpicture}
\caption{MR of a pseudo spin valve \ce{Mn2Ru$_x$Ga}(15)/\ce{Cu}(2.8)/[\ce{Co}(0.2)/\ce{Pd}(0.6)]$_6$/\ce{Ta}(\SI{3}{\nano\metre})  measured at various temperatures. The curves have been offset vertically for clarity.  The inset shows the temperature variation of the GMR contribution with a fit to $T^{0.5}$ dependence.}
\label{fig:gmr}
\end{figure}

%section on GMR
Finally we measured the magnetoresistance (MR) properties of the \ce{MRG}/\ce{Cu}/[\ce{Co}/\ce{Pd}] samples at different temperatures from \SIrange{2}{300}{\kelvin}. The MR was measured on unpatterned films in the current-in-plane configuration. A MR effect was cleared observed at \SI{2}{\kelvin}, and persists even at room temperature as shown in Fig. \ref{fig:gmr}. The observed MR is however quite low ($\sim \num{0.15}$) even at \SI{4}{\kelvin} which may be due to two effects: Firstly considering the transfer between separate deposition chambers for the \ce{MRG} and \ce{Cu}/[\ce{Co}/\ce{Pd}] layers, some interfacial contamination or oxidation of the \ce{Mn} can be expected.  Secondly, based on the results shown for the thickness dependence of the \ce{MRG} films, as discussed above, we find that the films are increasingly strained as the thickness of the film is reduced. This causes a variation in the spin-dependent transport properties and compensation of the two magnetic sub lattices, compared to the thicker films. Furthermore we assume that magnetic domains are present in the MRG film as in antiferromagnets; GMR is lost relatively quickly due to domain structuring and imperfect rotation of the magnetisation in the two electrodes, as evidenced by dispersed switching field range as shown in the electronic transport (Fig. \ref{fig:she}, and \ref{fig:gmr}).

\section{Conclusion}
\label{sec:conclusion}
We have shown above that the spin-dependent transport properties of \ce{Mn2Ru$_x$Ga} are tuneable with both the Ru concentration $x$ and strain. Recent \textit{ab inito} calculations \cite{galanakisJAP2013} while providing some insight into the electronic structure, does not give convincing arguments explaining the variation of the transport properties both with varying Ru concentration $x$ and strain. Above we have shown that for a Ru concentration $x\approx\num{0.7}$, which shows practically zero magnetization, the sign of the spontaneous Hall effect is reversed, indicating the reversal of the majority spin channel. Concurrently the spontaneous Hall angle is maximised which would imply a reduction in the carrier concentration and high spin polarisation that point towards a half metallic state. We also show that by varying the tetragonal distortion at a particular Ru composition, we can tune the compensation of the two Mn sub lattices to be at a relavant temperature regime at above or below room temperature. 
The initial demonstration of magnetoresistance in pseudo-spin-valves with an \ce{MRG} electrode indicates that while we are able to observe a MR effect, further understanding of the magnetic domain and micromagnetic structures are necessary for improving device performance. 
 
\section{Acknowledgements}
\label{sec:acknowledgements}
This work was supported by AMBER. KR acknowledges financial support from IFOX NMP3-LA-2010-246102. DB acknowledges financial support from IRCSET. We would also like to acknowledge Mario $\check{Z}$ic and Thomas Archer for their fruitful discussions.

\bibliography{transp}
\end{document}