\documentclass[reprint,aip,apl,floatfix,linenumbers,superscriptaddress]{revtex4-1}
\usepackage{graphicx}
\usepackage[product-units = power]{siunitx}
\usepackage[version=3]{mhchem}
\usepackage[utf8]{inputenc}
\usepackage{natbib}
%\usepackage{stackengine}

\DeclareSIUnit\BohrMagneton{$\mu_B$}
\DeclareSIUnit\formulaunit{f.u.}
\DeclareSIUnit\atomicunit{a.u.}
\DeclareSIUnit\arbunit{arb.unit}
\DeclareSIUnit\torr{Torr}

\begin{document}

%title, authors, affiliations
\title{Transport properties of cubic zero-moment ferromagnetic \ce{Mn2Ru$_x$Ga
} thin films}
\author{Naganivetha Thiyagarajah}
\author{Yong-Chang Lau}
\author{Karsten Rode}
\author{Davide Betto}
\author{Kiril Borisov}
\author{J. M. D. Coey}
\author{Plamen Stamenov}
\affiliation{CRANN, AMBER and School of Physics, Trinity College Dublin, 
Dublin 2, Ireland}

\date{\today}

%abstract needs to be re-written
\begin{abstract}
We have studied the spin-dependent transport properties of cubic \ce{Mn2Ru$_x$
 Ga} thin-films as function of the the Ru concentration, $x$ and the 
substrate induced strain. We observe large spontaneous Hall angles which are 
much larger than those observed in other $3d$ metals. In addition, we are 
able to tune the compensation of the two \ce{Mn} sublattics to a temperature 
at, above or below room temperature, by small changes to the tetragonal 
distortion. Having two handles on the zero moment half magnetic properties of 
\ce{Mn2Ru$_x$Ga} opens up the possibilities for using this new class of 
material in various spintronic devices. 

%We also %present the initial work on magnetoresistive devices based on 
pseudo-spin- %valves with \ce{Mn2Ru$_x$Ga} electrodes.


\end{abstract}
\maketitle

%Introduction 
\section{Introduction}
\label{sec:intro}

Cubic ferromagnetic Heusler compounds are a family of magnetic materials that 
often exhibit higher spin polarization at the Fermi level than binary 
ferromagnetic $3d$ alloys\cite{Graf2013}. Some of the materials are half- 
metals with a gap in the spin-polarized density of states for one spin band 
which should make them ideal candidates for spin-valves or MTJs\cite{PhysRevB.%
28.1745,Wang2009,Takahashi2011,Tsunegi2008}.  Since the prediction by van 
Leuken and de Groot in $1995$, of a half-metallic material with two in- 
equivalent magnetic sub-lattices whose moments cancel out \cite{PhysRevLett.50%
.2024}, researchers have worked on fabricating such a material. While 
electronic structure calculations predicted several such compounds\cite{ 
Wurmehl2006, Hu2012, Galanakis2006}, fabrication of such materials had failed 
\cite{Hu2012,PhysRevB.79.100406}. In $2014$, Kurt \textit{et. al.} reported 
the growth of thin films of \ce{Mn2Ru$_x$Ga} (\ce{MRG}), which was identified 
as a zero-moment ferrimagnet with high spin polarization, high $T_C$ up to \SI
{550}{\kelvin} and showed evidence of half-metallicity\cite{KurtPRL2014}. 
 Here we report on the temperature, composition and thickness dependent 
transport properties of \ce{MRG}, which are at or near compensation point ($0
. 6<x<1.1$). Addition of Ru to the cubic \ce{Mn2Ga} structure provides both 
states (\num{12}) and electrons (\num{8}). Based on the on the empirical 
Slater-Pauling rules, should result in perfect compensation for \ce{Mn2Ru$_{0
. 5}$Ga}. However the addition of Ru is likely to change both the shape and 
position of the Mn bands leading to a more complex behaviour of the magnetic 
and spin-dependent transport properties. In addition the tetragonal 
distortion ($c/a$) can also affect the band structure, hence we also look at 
strain as a possible control parameter in engineering the \ce{MRG} fully 
compensated half metallic system. 

%experiment 
\section{Experimental techniques}
\label{sec:exp_tech}

\ce{MRG} films of thickness \SIrange{4}{70}{\nano\metre} were grown on \ce{
MgO } (001) substrates by dc-magnetron sputtering at \SI{250}{\celsius} 
substrate temperature and base pressure \SI{2e-8}{\torr} in a Shamrock 
deposition system. The films were co-sputtered from a Mn2Ga target and Ru 
target, and the Ru composition was controlled by keeping the \ce{Mn2Ga} 
sputtering power fixed while varying that of Ru. The MRG films were capped 
with a  $\sim \SI{2 }{\nano\metre}$ \ce{Al2O3} layer to prevent oxidation.  
The crystal structure and lattice parameters were determined by $2\theta-
\theta$ and reciprocal space map (RSM) scans using a BRUKER D8 
diffractometer. In order to determine the \ce{Ru} concentration $x$, we 
deposited four samples with varying \ce{Mn2 Ga} target power along with a \ce{
Ru} film. The density and thickness of the samples were then measured using x-
ray reflectivity. Based on the measured density and lattice parameters of 
these 5 control samples, we establish a relation between the x-ray density 
and the \ce{Ru} concentration $x$ against which all the samples are 
calibrated. Magnetization measurements were made using a Quantum Design 
superconducting quantum interference device (SQUID) magnetometer. The 
transport measurements were conducted on unpatterned \ce{MRG } films in a 
physical properties measurement system (PPMS) for temperatures from \SIrange{
10}{400}{\kelvin}. The maximum applied magnetic fields, $\mu_0H$ , for the 
two systems were \SI{5}{\tesla} and \SI{14}{\tesla} respectively. L- edge X-
ray absorption and dichroism were also measured for the same set of samples 
to determine the spin and orbital moments. We found evidence of half- 
metalicity up to $x=0.7$, and the compensation of the \ce{Mn} $4a$ and $4c$ 
spin moments, resulting in a zero-moment half-metal in this region. \cite{
Betto2014}

%A summary of sample properties is provided in Table \ref{tab:samples}. We also %incorporated the \ce{MRG} as the hard layer into a pseudo-spin-valve with the %structure, \ce{MgO}/ \ce{MRG}(15)/\ce{Cu}(2.8)/[\ce{Co}(0.2)/\ce{Pd}(0.6)]$_6$ %/\ce{Ta}(\SI{3}{\nano\metre}) in order to investigate the spin dependent %transport. The \ce{MRG} layer was grown at \SI{250}{\celsius}, then cooled %down to room temperature, and was subsequently transferred to a different %deposition chamber for the \ce{Cu}/[\ce{Co}/\ce{Pd}] multilayer deposition. %Atomic force microscopy measurements of the \ce{MRG} film showed a roughness %of $\sim \SI{0.2}{\nano\metre}$, free of pinholes. 

%table with properties
%\begingroup
%\squeezetable
%\begin{table}
  %\caption{Summary of sample properties. The temperature at which full compensation occurs, $T_{comp}$ was defined by the temperature where $\partial \rho_{xy}/\partial T$ reaches its maximum.}
  %\begin{ruledtabular}
  %\begin{tabular}{c c c c c }
    %\ce{Ru} $x$ & $t$ &  $c/a-1$ & $M_s$ & $T_{comp}$ \\ 
		 %& \SI{}{\nano\metre} & \SI{}{\percent} & \SI{}{\BohrMagneton}  & \SI{}{\kelvin} \\
     %\hline
     %\num{0.62} & \num{70} & \num{2.07} & \num{-0.09} & \numrange[range-phrase = --]{100}{200} \\ 
%
%%NT059
     %\num{0.69} & \num{70} & \num{1.76} & \num{0.03} & \numrange[range-phrase = --]{200}{300} \\ %NT058
     %\num{0.73} & \num{70} & \num{1.83} & \num{0.07} & \numrange[range-phrase = --]{300}{360} \\ %NT063
		 %\num{0.77} & \num{70} & \num{1.92} & \num{0.09} & \num{> 360} \\ %NT055
		 %\num{1.09} & \num{70} & \num{1.82} & \num{0.07} & \num{> 360} \\ %NT046
     %\num{1.12} & \num{70} & \num{1.84} & \num{0.07} & \num{387} \\ %NT072 
		 %\num{1.01} & \num{34} & \num{1.92} & - & \num{335} \\ %NT073
		 %\num{0.98} & \num{16} & \num{2.24} & - & \num{280} \\ %NT074
     %\num{1.09} & \num{8} & \num{2.90} & - & \num{214} \\ %NT075
		 %\num{1.07} & \num{4} & \num{3.60} & - & \num{< 10} \\ %NT076	 
  %\end{tabular}
  %\end{ruledtabular}
  %\label{tab:samples}
%\end{table}
%\endgroup

%results section needs to be expanded - more discussion
\section{Results and Discussion}
\label{sec:results_discussion}

The crystal structure of the cubic \ce{MRG} films with different thickness 
and compositions were probed using $2\theta-\theta$ x-ray diffraction (XRD). 
The out-of-plane lattice parameter, $c$, is between \SI{0.598}{\nano\metre} 
and \SI{0.618}{\nano\metre}, depending on the \ce{Ru} concentration and film 
thickness. We find that  $c$ increases exponentially with reducing film 
thickness. The in-plane lattice parameter, $a$, determined from reciprocal 
space maps was found to be \SI{0.596}{\nano\metre} for all samples, which is 
precisely matched to that of the \ce{MgO} substrate ($\sqrt{2} a_0\left(\ce{
MgO} \right)  = \SI{0.5956}{\nano\metre}$). This confirms the cubic nature of 
the \ce{MRG} films with a slight tetragonal out-of-plane distortion ($c/a-1$ 
between \num{1 .8}\% and \num{3.6}\%).

SQUID magnetometry showed clear out-of-plane anisotropy in all the samples we 
have studied.  A small soft in-plane component was also present. As the \ce{Ru
} concentration is reduced from $x=\num{1.09}$, the magnetization reduces, 
until it falls practically to zero (\SI{12}{\kilo\ampere\per\metre} or \SI{0.
07}{\BohrMagneton\per\formulaunit}) at $x= \num{0.68}$. We can attribute this 
to the almost perfect compensation of the two \ce{Mn} sub-lattices at room 
temperature. On further reduction of Ru the magnetization again increases, 
coincident with the reversal in sign in the room temperature spontaneous Hall 
effect (SHE) measurements. 

%fig 4 on strain/thickness goes here
\begin{figure}
\includegraphics[width=1.0\columnwidth]{Transport-Fig0.pdf}
\caption{(a) Variation of compensation temperature with the thickness of \ce{ 
MRG} film of same \ce{Ru} concentration, given by the derivative of the 
resistance w.r.t temperature. The compensation temperature shifts to lower 
temperatures with decreasing thickness. (b) Extracted coercive field and (c) 
spontaneous Hall angle as a function of temperature for samples with the same 
\ce{Ru} concentration ($x \sim \num{1.0}$) and various thickness from \SIrange
 {70}{4}{\nano\metre}.}
\label{fig:strain}
\end{figure}

It has been predicted that the magnetization may depend strongly on the 
lattice distortion \cite{Mario2014} since this would have an effect on the 
interaction between neighbouring  atoms. We prepared \ce {MRG} samples of 
different thickness from \SI{70}{\nano\metre} down to \SI{4}{\nano\metre} and 
measured their SHE response at different temperatures from \SI{400}{\kelvin} 
to \SI{4}{\kelvin} in the PPMS. As mentioned above the out-of-plane lattice 
parameter. $c$, increased exponentially with reducting sample thickness 
allowing us to have a control of the slight tetragonal distortion of the 
samples with a similar Ru composition. By plotting the derivative of the Hall 
resistance w.r.t temperature, $\delta R_{ XY}/\delta T$, as shown in Fig. \ref
{fig:strain}(a), it can be seen that this compensation temperature, $T_{comp}$
 shifts to lower temperatures as the thickness of the \ce{MRG} is reduced. It 
is worth noting that the compensation temperature varies with both the Ru 
content and strain. Since the compensation is achieved by the cancelling out 
of the moment of the two inequivalent \ce{Mn} sub-lattices, this shift in 
compensation temperature may be due to the slightly different temperature 
dependence of the two sub-lattices. From the SHE measurements we also note 
that as the magnetization approaches zero the coercivity clearly diverges (
since $H_c=2K_u/M_s$) as shown in Fig. \ref{fig:strain}(b). We also extracted 
the spontaneous Hall angle (SHA) (defined as $\rho_{xy}$/$\rho_{xx}$) as 
shown in Fig. \ref{fig:strain} (c). We qualify the sign reversal of the SHE 
signal with the change in sign of the SHA at the compensation temperature. We 
note only slight variations in the absolute value of the SHA with changing 
temperature. In the case of the thinnest sample with of \SI{4}{\nano\meter}, 
which also has the largest $c/a$ ratio, the curie temperature is shifted to 
much lower temperatures which is observed from the near zero SHA with an 
increase only close to \SI{10}{\kelvin}. 

%figure3 - SHA vs x (all samples)
\begin{figure}
\includegraphics[width=1.0\columnwidth]{Transport-Fig1.pdf}
\caption{Evolution of the spontaneous Hall angle (SHA) as a function of \ce{Ru
} composition, $x$, extracted from SHE measurements. The lines are a linear 
fit of the data sets.}
\label{fig:SHAvx}
\end{figure}

Fig. \ref{fig:SHAvx} shows the evolution of the SHA with varying Ru 
concentration, $x$, for $0.6<x<1.1$. From the temperature dependent SHE 
measurements above we have seen that the SHA is mostly constant over a 
temperature range lower than $T_c$, hence we have taken the average SHA over 
the measured temperature range for all samples.  Fig. \ref{fig:SHAvx} also 
plots the variation in the longitudinal resistivity, $\rho_{xx}$ and the Hall 
resistivity,$\rho_{xy}$ for the same samples. As can be seen the $\rho_{xx}$ 
is mostly constant over the range of $x$.  The recorded SHA ($\sim \num{5}\%$)
 for samples up to $x=0.7$, which are in the half metalic region, are about a 
magnitude larger than those reported for other 3d ferromagnets at room 
temperature (\numrange{0.2}{0.3}\%)\cite{dorleijn1976} and comparable to SHA 
recorded for amorphous rare earth transition metal alloys\cite{Kim2001}. A 
high SHA is indicative of much lower carrier concentrations and a high spin 
polarization. The increase in SHA with decreasing Ru content implies that the 
large SHA is not due to increased spin-orbit scattering from the addition of 
Ru. 

%figure3 - SHA vs c/s (all samples)
\begin{figure}
\includegraphics[width=1.0\columnwidth]{Transport-Fig2.pdf}
\caption{Evolution of the spontaneous Hall angle (SHA) as a function of the 
tetragonal distorsion ($c/a$ ratio) extracted from SHE measurements for 
samples with their SHA translated to a virtual $x=1.0$ Ru concentration.}
\label{fig:SHAvca}
\end{figure}

In order to look more close of the effect of the tetragonal distortion on the 
SHA, we also plot the evolution of SHA with $c/a$ ratio. As we have seen in 
Fig. \ref{fig:SHAvx} there is a large variation in SHA with the Ru content (
from \numrange{1}{6} \%). Therefore we have extrapolated the SHA for a 
virtual sample with $x=1$ based on the dependence in Fig. \ref{fig:SHAvx}. 
Fig. \ref{fig:SHAvca} shows the evolution of the SHA with increasing $c/a$. 
There is no clear dependence of the SHA on the tetragonal distortion. While a 
slight change in the $c/a$ changes drastically magnetic compensation 
temperature, it appears to have a negligible effect on the the spin 
polarization.

%%fig 5 GMR goes here 
%\begin{figure}
%\includegraphics[width=1.0\columnwidth]{Transport-Fig3.pdf}
%\caption{MR of a pseudo spin valve \ce{Mn2Ru$_x$Ga}(15)/\ce{Cu}(2.8)/[\ce{Co}( %0.2)/\ce{Pd}(0.6)]$_6$/\ce{Ta}(\SI{3}{\nano\metre})  measured at various %temperatures. The curves have been offset vertically for clarity.  The inset %shows the temperature variation of the GMR contribution with a fit to $T^{0.5 %}$ dependence.}
%\label{fig:gmr}
%\end{figure}
%
%%section on GMR
%Finally we measured the magnetoresistance (MR) properties of the \ce{MRG}/\ce{ %Cu}/[\ce{Co}/\ce{Pd}] samples at different temperatures from \SIrange{2}{300}{ %\kelvin}. The MR was measured on unpatterned films in the current-in-plane %configuration. A MR effect was cleared observed at \SI{2}{\kelvin}, and %persists even at room temperature as shown in Fig. \ref{fig:gmr}. The %observed MR is however quite low ($\sim \num{0.15}$) even at \SI{4}{\kelvin} %which may be due to two effects: Firstly considering the transfer between %separate deposition chambers for the \ce{MRG} and \ce{Cu}/[\ce{Co}/\ce{Pd}] %layers, some interfacial contamination or oxidation of the \ce{Mn} can be %expected.  Secondly, based on the results shown for the thickness dependence %of the \ce{MRG} films, as discussed above, we find that the films are %increasingly strained as the thickness of the film is reduced. This causes a %variation in the spin-dependent transport properties and compensation of the %two magnetic sub lattices, compared to the thicker films. Furthermore we %assume that magnetic domains are present in the MRG film as in %antiferromagnets; GMR is lost relatively quickly due to domain structuring %and imperfect rotation of the magnetisation in the two electrodes, as %evidenced by dispersed switching field range as shown in the electronic %transport (Fig. \ref{fig:she}, and \ref{fig:gmr}).

\section{Conclusion}
\label{sec:conclusion}
We have shown above that the spin-dependent transport properties of \ce{Mn2Ru$
_x$Ga} are tuneable with both the Ru concentration $x$ and strain. Recent 
\textit{ab inito} calculations \cite{galanakisJAP2013} while providing some 
insight into the electronic structure, does not give convincing arguments 
explaining the variation of the transport properties both with varying Ru 
concentration $x$ and strain. Above we have shown that for a Ru concentration 
$x\approx\num{0.7}$, which shows practically zero magnetization, the sign of 
the spontaneous Hall effect is reversed, indicating the reversal of the 
majority spin channel. Concurrently the spontaneous Hall angle is maximised 
which would imply a reduction in the carrier concentration and high spin 
polarisation that point towards a half metallic state. We also show that by 
varying the tetragonal distortion at a particular Ru composition, we can tune 
the compensation of the two Mn sub lattices to be at a relavant temperature 
regime at above or below room temperature. 
%The initial demonstration of magnetoresistance in pseudo-spin-valves with an %\ce{MRG} electrode indicates that while we are able to observe a MR effect, %further understanding of the magnetic domain and micromagnetic structures are %necessary for improving device performance. 
 
\section{Acknowledgements}
\label{sec:acknowledgements}
This work was supported by Science Foundation Ireland through AMBER, and from 
grant 13/ERC/I2561. KR acknowledges financial support from the European 
Community's Seventh Framework Programme IFOX, NMP3-LA-2010-246102. DB 
acknowledges financial support from IRCSET. The authors would like to thank 
H. Kurt, M. \v{Z}ic and T. Archer for fruitful discussions.

\bibliography{transp}
\end{document}